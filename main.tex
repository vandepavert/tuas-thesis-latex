\documentclass[a4paper]{report}
\usepackage[utf8]{inputenc}
\usepackage[english]{babel}
\usepackage{subfiles}
\usepackage{blindtext}
\usepackage{amsmath}
\usepackage{mathtools}
\usepackage{amssymb}
\usepackage{xifthen}
\usepackage{appendix}
\usepackage{lastpage}
\usepackage{csquotes}
\usepackage{verbatim}


% DEBUGGING: show the margins so that the indentation is clearly shown.
%----------------------------------------------
\usepackage[showframe]{geometry}


% Format paragraph
%----------------------------------------------
\usepackage{titlesec}
%\titlespacing{\paragraph}{left margin}{space before (vertical)}{space after (horizontal)}
\titlespacing{\paragraph}{0pt}{0.25\baselineskip}{0em}


% Format tables and long multi-page tables
%----------------------------------------------
\usepackage{longtable}
\usepackage{adjustbox}
\usepackage{caption}
\captionsetup[table]{skip=5pt}


% Scaled images to stay within margins
%----------------------------------------------
\usepackage{graphicx}
\graphicspath{{images/}{../images/}}
\newcommand{\maxwidth}[2][\linewidth]{% \maxwidth[<max hlen>]{<stuff>}
  \setbox0=\hbox{#2}% Store image in box0
  \ifthenelse{\dimtest{\wd0}>{#1}}%
    {\resizebox{#1}{!}{#2}}% Scale object to fit within \linewidth
    {\usebox0}% Use original (unaltered in width) object
}


% Vector grapics
%----------------------------------------------
\usepackage{tikz}
\usetikzlibrary{
  arrows.meta,
  calc,
  chains,
  positioning,
  3d,
  shapes.geometric,
  intersections,
}
\newcommand\ppbb{path picture bounding box}

\makeatletter
\tikzset{suppress join/.code={\def\tikz@after@path{}}}
\makeatother

% https://tex.stackexchange.com/questions/482057/extending-anchors-in-tikz
\def\findmid#1#2#3{($(#2)!#1!(#3)$)}


% Create illustrations for network protocol specifications and anything else that utilizes fields of data
%----------------------------------------------
\usepackage{bytefield}


% Calculations are done in Lua, Python or Expl3, not in LaTeX
%----------------------------------------------
\usepackage{expl3}


% Initialise nomenclature to show List of Symbols and Notations
%----------------------------------------------
\usepackage{nomencl}
\makenomenclature

\renewcommand{\nomname}{List of Symbols and Notations}
\renewcommand{\nompreamble}{Polynomials have their respective degree shown in brackets.}


% Create nomenclature groups
%----------------------------------------------
\usepackage{etoolbox}
\renewcommand\nomgroup[1]{%
  \item[\bfseries
  \ifstrequal{#1}{A}{Variables}{%
  \ifstrequal{#1}{B}{Polynomials}{%
  \ifstrequal{#1}{C}{Number Sets}{}}}%
]}
 

% This will add the degrees of the polynomials
%----------------------------------------------
\newcommand{\nomunit}[1]{%
  \renewcommand{\nomentryend}{\hspace*{\fill}(#1)}}
  % fixme: fill hspace with dots like \tableofcontents, \listoffigures and \listoftables


% Create glossary of abbreviations
%----------------------------------------------
\usepackage[acronym,toc]{glossaries}
%\makeglossaries


% Add BibTex references
%----------------------------------------------
\usepackage[backend=biber, style=apa, sorting=nty]{biblatex}
\newbibmacro*{bbx:parunit}{%
  \ifbibliography
    {\setunit{\bibpagerefpunct}\newblock
     \usebibmacro{pageref}%
     \clearlist{pageref}%
     \setunit{\adddot\par\nobreak}}
    {}}

\renewbibmacro*{doi+eprint+url}{%
  \usebibmacro{bbx:parunit}% Added
  \iftoggle{bbx:doi}
    {\printfield{doi}}
    {}%
  \iftoggle{bbx:eprint}
    {\usebibmacro{eprint}}
    {}%
  \iftoggle{bbx:url}
    {\usebibmacro{url+urldate}}
    {}}

\renewbibmacro*{eprint}{%
  \usebibmacro{bbx:parunit}% Added
  \iffieldundef{eprinttype}
    {\printfield{eprint}}
    {\printfield[eprint:\strfield{eprinttype}]{eprint}}}

\renewbibmacro*{url+urldate}{%
  \usebibmacro{bbx:parunit}% Added
  \printfield{url}%
  \iffieldundef{urlyear}
    {}
    {\setunit*{\addspace}%
     \printtext[urldate]{\printurldate}}}
\addbibresource{references.bib}


% Cross document referencing in standalone documents
%----------------------------------------------
\usepackage{xr}
%\usepackage{xr-hyper} % in case of also loading hyperref.
\makeatletter
\newcommand\longempty{}%
\newcommand\DoIfAndOnlyIfStandAlone{%
  \ifx\document\longempty
    \expandafter\@gobble
  \else
    \expandafter\@firstofone
  \fi
}%
\makeatother
\usepackage[noabbrev]{cleveref}


% copy matter parts from book class
%----------------------------------------------
\makeatletter
    \newcommand\frontmatter{%
        \cleardoublepage
        \pagenumbering{roman}
    }
    
    \newcommand\mainmatter{%
        \cleardoublepage
        \pagenumbering{arabic}
    }
    
    \newcommand\backmatter{%
      \if@openright
        \cleardoublepage
      \else
        \clearpage
      \fi
    }   
\makeatother
%----------------------------------------------


\title{Reed-Solomon Encoding and Decoding \\\large A Visual Representation}
\author{León van de Pavert}
\date{30 June 2011}
 
\begin{document}
    % FRONT 
    %----------------------------------------------
    \frontmatter
    
      \maketitle

      % Introductory chapters

      % Abstract
      \subfile{sections/abstract.tex}

      % Acknowledgements
      \subfile{sections/acknowledgements.tex}

      % Table of Contents
      \tableofcontents

      % List of Figures
      \listoffigures
      \addcontentsline{toc}{chapter}{List of Figures}

      % List of Tables
      \listoftables
      \addcontentsline{toc}{chapter}{List of Tables}

      % List of Symbols and Notations
      \mbox{}

\nomenclature[A,01]{$n$}{length of codeword}
\nomenclature[A,02]{$k$}{number of data symbols}
\nomenclature[A,03]{$d$}{distance}
\nomenclature[A,04]{$d_{min}$}{minimum distance}
\nomenclature[A,05]{$t$}{number of correctable errors}
\nomenclature[A,06]{$l$}{number of detectable errors}

\nomenclature[B, 01]{$g(x)$}{generator \nomunit{$n-k$}}
\nomenclature[B, 02]{$p(x)$}{error check \nomunit{$n-k-1$}}
\nomenclature[B, 03]{$h(x)$}{parity check \nomunit{$k$}}
\nomenclature[B, 04]{$i(x)$}{information \nomunit{$k-1$}}
\nomenclature[B, 05]{$c(x)$}{code-word \nomunit{$n-1$}}
\nomenclature[B, 06]{$c'(x)$}{received code-word \nomunit{$n-1$}}
\nomenclature[B, 07]{$c'_r(x)$}{corrected code-word \nomunit{$n-1$}}
\nomenclature[B, 08]{$s(x)$}{syndrome \nomunit{$n-k-1$}}
\nomenclature[B, 09]{$e(x)$}{error \nomunit{$n-1$}}

\nomenclature[C, 01]{$\mathbb{N}$}{set of natural numbers \{0, 1, 2, ...\}}
\nomenclature[C, 02]{$GF(q)$}{Galois field or finite field where $q \in \mathbb{N}$}

      \printnomenclature
      \addcontentsline{toc}{chapter}{List of Symbols and Notations}

      % List of Abbreviations
      %\newacronym{adc}{ADC}{Analogue-to-digital converter}
\newacronym{bch}{BCH}{A class of codes named after Bose, Ray-Chaudhuri and Hocquenghem}
\newacronym{bec}{BEC}{Binary erasure channel}
\newacronym{bsc}{BSC}{Binary symmetric channel}
\newacronym{ecc}{ECC}{Error-correcting code}
\newacronym{fec}{FEC}{Forward error correction}
\newacronym{mds}{MDS}{Maximum distance separable}
\newacronym{ml}{ML}{Maximum likelihood}
\newacronym{mld}{MLD}{Maximum likelihood decoding}
\newacronym{rs}{RS}{A class of codes named after Reed and Solomon}

      %\printglossary[type=acronym, title=Abbreviations, toctitle=Abbreviations]
      %\addcontentsline{toc}{chapter}{List of Abbreviations}


    % MAIN 
    %----------------------------------------------
    \mainmatter
        
      \subfile{sections/introduction.tex}
      \subfile{sections/coding-theory-basics.tex}
      \subfile{sections/linear-block-codes.tex}
      \subfile{sections/visualisation.tex}
      \subfile{sections/summary-and-conclusion.tex}


    % APPENDIX 
    %----------------------------------------------
    %\appendix
    
    
    % BACK 
    %----------------------------------------------
    \backmatter

      \chapter*{References}
      \addcontentsline{toc}{chapter}{References}
      \printbibliography[heading=none]

\end{document}

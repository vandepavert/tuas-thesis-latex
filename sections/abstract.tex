\documentclass[../main.tex]{subfiles}
 
\begin{document}

The capacity of a binary channel is increased by adding extra bits to this data. This improves the quality of digital data.The process of adding redundant bits is known as channel encoding.

In many situations, errors, are not distributed at random but occur in bursts. For example, scratches, dust or fingerprints on a compact disc (CD) introduce errors on neighbouring data bits. Cross interleaved Reed-Solomon Codes (CIRC) are particular well-suited for detection and correction of burst errors and erasures. Interleaving redistributes the data over many blocks of code. The double encoding has the first code declaring the erasures. The second code corrects them.

The purpose of this thesis is to present Reed-Solomon error correction codes in relation to burst errors. In particular, this thesis visualises the mechanism of cross-interleaving and its ability to allow for detection and correction of burst errors.

\paragraph{}
KEYWORDS: Coding theory, Reed-Solomon code, burst errors, cross-inter\-leaving, compact disc

\paragraph{}
Bachelor's Thesis \textbar\space Abstract \\
Turku University of Applied Sciences \\
Degree Programme in Information Technology \\
Spring 2011 \textbar\space \pageref{LastPage} pages \\
Instructor: Hazem Al-Bermanei \\

\end{document}
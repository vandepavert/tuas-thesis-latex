\documentclass[../main.tex]{subfiles}
\setcounter{chapter}{2}

\begin{document}

    \chapter{Linear Block Codes}

    \section{Hamming weight, Minimum Distance and Code Rate}
    The Hamming weight $w_H(x)$ of a codeword or vector $x$ is defined as the amount of non-zero elements or vector coordinates, which ranges from zero to length $n$ of said codeword.

    \begin{equation*}
        w_H(x) = \sum_{j=0}^{n-1} w_H(x_j), \text{where } w_H(x_j) = \begin{cases} 0, &\text{if } x_j = 0 \\ 1, &\text{if } x_j \neq 0 \end{cases}
    \end{equation*}

    \noindent
    The Hamming distance $d_H(x,y)$ between two codewords or vectors $x$ and $y$ is defined as amount of elements or coordinates where $x$ and $y$ differ.

    \begin{equation*}
        d_H(x,y) = \sum_{j=0}^{n-1} w_H(x_j+y_j), \text{where } w_H(x_j+y_j) = \begin{cases} 0, \text{if } x_j=y_j \\ 1, \text{if } x_j \neq y_j \end{cases}
    \end{equation*}

    \begin{equation*}
        d_H(x,y) =  w_H(x,y)
    \end{equation*}

    \noindent
    The minimum distance $d_min$ of code $C$ is the minimum distance between two different codewords. The minimum distance for linear codes is equal to the minimum weight \autocite{bossert1999channel}. However, a codeword containing only zeros and therefore having a distance of zero is disregarded as the minimum distance cannot be zero.


    Let $x,y$ be codewords in code $C$. A received vector, which is the sent vector $x$ in $C$, plus error vector $e$ can only be corrected if the distance between any other codeword $y$ in $C$ fulfill

    \begin{equation*}
        d_{min}(x, x+e) < d_{min}(y, x+e) \text{ or } w_{min}(e) < w_{min}(x+y+e).
    \end{equation*}

    \noindent
    Therefore

    \begin{equation*}
         w_{min}(e) \leq \frac{d-1}{2},
    \end{equation*}

    \noindent
    where $d$ is the distance. This is written as

    \begin{equation*}
        t \leq \frac{d-1}{2} \text{or } d \geq 2t+1,
    \end{equation*}

    \noindent
    where $t$ is the amount of errors that can be corrected.


    A q-ary code of lenght $n$, with $k$ codewords, and a minimum distance $d_{min}=d(C)$, is called a $(n,k,d)$ code. The code rate is defined as

    \begin{equation*}
        R_c=\frac{log_qk}{n}.
    \end{equation*}

    \noindent
    Linear (n,k)-codes use n symbols to send k message symbols. Its code rate is:

    \begin{equation*}
        R_c=\frac{k}{n},
    \end{equation*}

    \noindent
    where $k$ is the number of bits and $n$ is the length of the code. If rate $R$ less than capacity $C$ then the code exists, but if rate $R$ is larger than capacity $C$, the error probability is 1 and the length of the codeword becomes infinite.

    \section{Singleton Bound}


    \section{Maximum-likelihood Decoding}


    \section{Hamming Codes}


    \section{Syndrome Decoding}


    \section{Cyclic Codes}


    \section{BCH Codes}


    \section{Generating BCH Code}


    \section{Decoding BCH Code}


\end{document}



